\documentclass{article}
\usepackage[hyphens,spaces,obeyspaces]{url}
\usepackage{pgfplots}
\usepackage{tikz}
\usepackage{graphicx}
\graphicspath{ {./images/} }
\usepackage{amsmath}
\usepackage{amsthm}
\usepackage{amssymb}
\usepackage{mathabx}
\usepackage{amsfonts}
\usepackage{enumitem}
\usepackage{algorithm}
\usepackage{algorithmic}
\graphicspath{ {./images/} }
\usetikzlibrary{shapes}
\usepgfplotslibrary{polar}
\usetikzlibrary{decorations.markings}
\usetikzlibrary{backgrounds}
\pgfplotsset{every axis/.append style={
                    axis x line=middle,    % put the x axis in the middle
                    axis y line=middle,    % put the y axis in the middle
                    axis line style={<->,color=blue}, % arrows on the axis
                    xlabel={$},          % default put x on x-axis
                    ylabel={$},          % default put y on y-axis
            }}
\newcommand{\numpy}{{\tt numpy}}    % tt font for numpy
\usepackage[utf8]{inputenc}
\usepackage{titlesec}

\newtheorem{theorem}{Theorem}
\newtheorem{lemma}[theorem]{Lemma}
\newtheorem{proposition}[theorem]{Proposition}
\newtheorem{corollary}[theorem]{Corollary}
\newtheorem{conjecture}{Conjecture}
\newtheorem{definition}{Definition}
\theoremstyle{remark}
\newtheorem{example}{Example}
\newtheorem{remark}[example]{Remark}

\newcommand{\bigsection}[1]{
  \titleformat*{\section}{\LARGE\bfseries}
  \section*{#1}
  \titleformat*{\section}{\large\bfseries} % Reset to the original format
}

\title{Theory of statistics and probability}
\author{MinSeok Song}
\date{}
\usepackage{pgfplots}
\pgfplotsset{compat=1.18}
\begin{document}
\maketitle
\bigsection{Statistics}
The dichotomy of Bayesian and Frequentist approach is mostly a matter of division based on foundational philosophical differences, and in some situation is not necessarily natural; each instance can be formally simply seen as a choice of analysis. The dichotomy has mainly developed based on two philosophical school of thoughts:
 subjective belief (via prior distribution) vs. long-run frequency (via sampling distribution).
Frequentist
\begin{itemize}
\item They avoid making probabilistic claims about model parameters and hypothesss. Instead, they describe the behavior of statistics and procedures over many hypothetical repeated samples.
\item They use Statistics derived from data, and use deterministic approach for inference, with methods like hypothesis testing and confidence intervals to draw conclusions about population parameters based on sample data.
\item examples: T-test, linear regression, etc
\end{itemize}

Bayesian
\begin{itemize}
\item They assign prior beliefs (distribution function) on parameters of model.
\item They use probabilistic argument on specific hypothesis or parameter values.
\item examples: MCMC, Bayesian hierarchical modeling, etc
\end{itemize}

\subsection*{data science}[list of topics that will be discussed]
\begin{enumerate}
\item Preparation of data
\begin{itemize}
\item handling missing data (imputation, etc)
\item transformation
\item Box-Cox transformation
\item Outliers
\end{itemize}

\item Interpret descriptive statistics about data
\begin{itemize}
\item plots
\item t-distribution
\item F-distribution
\end{itemize}

\item Modeling (parametric, nonparametric)
\begin{itemize}
\item ANOVA
\item Ensemble methods
\item Boosting
\item AIC-BIC
\item Regularization
\end{itemize}

\item Predict based on model
\begin{itemize}
\item Confidence/Prediction intervals
\item Real-time prediction
\end{itemize}

\item Inference on statistics
\begin{itemize}
\item Hypothesis testing
\item p-value
\item Type I/II error
\item Causal inference
\end{itemize}

\item Evaluation of models
\begin{itemize}
\item Cross-validation
\end{itemize}

\item Communication of models
\end{enumerate}

\section{Preparation of Data}
\subsection{Handling Missing Data}
\subsubsection{Types of Missing Data}
\begin{itemize}
\item MCAR
\item MAR
\item MNAR

\end{itemize}

\subsection{Outliers}

\section{Inference on statistics}
\subsection{p-value}
\begin{definition}
Given some hypothesis, the p-value is the probability of obtaining test results at least as extreme as 
the result actually observed, under the assumption that the null hypothesis is correct.
\end{definition}
\begin{itemize}
\item If p-value is less than significance level, we reject the null hypothesis. Significance 
level is a threshold we set to decide when we have enough evidence to reject the null hypothesis in favor of the alternative hypothesis.
\item In statistics, we do not quite "prove" that the hypothesis is true. We can only reject the false statements or assertions (the theory of falsification, proposed by Karl Popper). 
\end{itemize}

\subsection{t-distribution}
\begin{definition}
t-distribution is defined as $\frac Z{\sqrt{\frac Vk}}$ where $Z \perp V$, $Z\sim N(0,1)$, and $V\sim \chi_k^2$.
\end{definition}
\begin{itemize}
\item The t-distribution has the thicker tails than the normal distribution.
\item Note that the $\chi_k^2$ distribution has mean $k$ and variance $2k$. Hence as $k$ gets larger the t-distribution 
looks like standard normal distribution due to CLT. 
\item In multiple regression setting, $\frac{\frac{\hat\beta_1-\beta_1}{\frac \sigma{\sqrt{\sum_i(X_i-\bar X)^2}}}}{\sqrt{(n-2)\cdot\frac{\hat \sigma^2}{\sigma^2}/(n-2)}}\sim t_{n-2}$ under normal assumption of residual.
\subsubsection{Chi-squared distribution}
\begin{itemize}
\item It is possible to have $f(X)$ and $g(X)$ be independent. The key example concerns chi-distribution: Say $\theta$ is uniformly distributed in $[0,2\pi]$. 
Let us say $R^2$ follows $chi$-squared distribution with degree 2: this acts like a distance square in the 2d space. Then $X= r\cos(\theta)$ and $Y=r\sin(\theta)$ are independent.
\item The definition of chi-square looks contrived, in particular compared to normal distribution, but we use it a lot in the context of sum of squares.
\end{itemize}
\item Under moment assumption, it's approximately true for large $n$ via CLT.
\end{itemize}

\subsection{Prediction interval, Confidence interval}
\begin{itemize}
\item In the context of linear regression, a \textbf{confidence interval} provides a range of values within which we expect the true regression coefficient (e.g., $ \beta_0 $ or $ \beta_1 $) to lie with a certain level of confidence. This reflects our uncertainty about the 
value of the coefficient based on the data we have observed.
\item On the other hand, a \textbf{prediction interval} provides a range of values within which a new observation $ y $ (given a particular $ x $) is expected to fall with a certain level of confidence.
 We have $y=\beta_0+\beta_1 x+\epsilon$. Note the additional uncertainty introduced by the random error $\epsilon$.
\end{itemize}

\subsection{Fisher information}
\begin{definition}
Fisher information is defined by $\mathcal{I}(\theta)=E[(\frac \partial{\partial\theta}\log f(X;\theta))^2|\theta]$ for appropriately regular $f$.
\end{definition}

\begin{itemize}
\item Note that $E[\frac{\partial}{\partial\theta}\log f(X;\theta)|\theta]=0$.
\item Think of this as how much (the logarithm of) likelihood varies in the vicinity of $\theta$.
\item Large variation of likelihood we have more "information." (for example, if a deviation causes a significant drop in likelihood, 
it means that the data is most probable under the original $\theta$ value.)
\end{itemize}

\begin{theorem} [Cramer-Rao bound]
Say $\hat\theta(X)$ is an unbiased estimator. We have $Var(\hat\theta)\geq \frac 1{\mathcal{I}(\theta)}$.
\end{theorem}

\begin{itemize}
\item This says that as the fisher information gets larger, we have better precision.
\item There is a similar result for biased estimator.
\end{itemize}

\bigsection{Probability Theory}
\begin{definition}
    \textbf{Distribution Function (F)}: A function \( F: \mathbb{R} \rightarrow [0,1] \) that:
    \begin{enumerate}
        \item Is non-decreasing.
        \item Is right-continuous.
        \item Limits to 0 as \( x \rightarrow -\infty \) and 1 as \( x \rightarrow \infty \).
    \end{enumerate}
    
    \textbf{Random Variable (X)}: A measurable function \( X: \Omega \rightarrow \mathbb{R} \) where:
    \begin{enumerate}
        \item \( \Omega \) is a sample space equipped with a probability measure.
        \item \( \mathbb{R} \) is the real line equipped with the Lebesgue measure.
    \end{enumerate}
    
    \textbf{Probability Measure (P)}: A measure defined on a sample space \( \Omega \) whose total measure is 1.
    
    \textbf{Distribution (D)}: Given a random variable \( X \), its distribution is the measure induced by \( X \) on \( \mathbb{R} \).
    
    \textbf{Density Function (f)}: For a random variable \( X \), its density function \( f \) is any measurable function that satisfies:
    \[ \Pr(X \in A) = \int_A f \, d\mu \]
    for every measurable set \( A \subset \mathbb{R} \), where \( \mu \) is the Lebesgue measure.
    \end{definition}
    
There are several theorems/properties that look trivial but are not really trivial.
\begin{itemize}
\item The Skorokhod Representation Theorem 
    states that for every CDF, there exists a canonical probability space and a random variable on that space with the given CDF.
\item Kolmogorov's Existence threorem states that given finite dimensional sets of distribution, under some conditions, there exists a canonical probability space with random variables whose 
corresponding distributions coincide with our original distributions.
\item The existence of distribution function does not guarantee the existence of density function. By Radon-Nikodym, there is a nice characterization when this happens: when distribution (which is a measure defined on $\mathbb{R})$ corresponding to distribution function is absolutely continuous with respect to
 Lebesgue measure.
\item The existence of distribution function guarantees the existence of distribution, and vice versa.
\item However, the corresponding distribution always exists (also called generalized function). 
\item The previous discussion illustrates that Kolmogorov's existence theorem is more general than Skorokhod Representation theorem.
\end{itemize}
Throughout the theory of probability, we usually assume that the probability measure is complete, because of the following proposition.
\begin{proposition}
    Let us assume that probability measure is complete. Let $X_t=Y_t$ almost everywhere for every $t\geq 0$.
    Then there exists $\tilde{Y}_t=Y_t$ almost everywhere for every $t$ such that $X_t(w)=\tilde{Y}_t(w)$ for every $w\in P$ for every $t$ for 
    measurable set $P$ whose measure is 1.
\end{proposition}

\subsection*{Covariance, Independence}
\begin{itemize}
\item Covariance only gives the information about linear relationship between two random variables. It has a limit of capturing non-linear relationship.
\item Non-linear activation function in deep neural network is an example of such an attempt to leverage non-linearities.
\item Further, as correlation gives only the linear relationship, zero correlation does not necessarily imply independence.
\item Independence does not necessarily mean that, in a strict sense, two variables have no "relationsihps." It simply means
 that the information \textbf{about the value of the random variable} does not give any information of \textbf{the value of the other random variable}. For example, we may have $Y=X^2$ where $X=-1$ or $1$.
\item Also note that two random variables may have different mappings while having the same distribution: standard example is $1-X$ and $X\in [0,1]$.
\end{itemize}

\subsection*{Distribution function}
\begin{itemize}
\item If we were to calculate the density function of $X+Y$, we may use convolution.
\item If we were interested in $\frac XY$, we may use transformation and integrate out. 
\item Another analytic approach is to use characteristic functions.
\item We can use Monte Carlo method numerically.

\end{itemize}

\subsection*{Convergence}

\subsection*{Markov Chain}

\subsection*{Martingale}

\subsection*{Brownian Motion}

\subsection*{Relationship with PDE}

\subsection*{Relationship with Complex Analysis}
\end{document}