\documentclass{article}
\usepackage[hyphens,spaces,obeyspaces]{url}
\usepackage{pgfplots}
\usepackage{tikz}
\usepackage{graphicx}
\graphicspath{ {./images/} }
\usepackage{amsmath}
\usepackage{amsthm}
\usepackage{amssymb}
\usepackage{mathabx}
\usepackage{amsfonts}
\usepackage{enumitem}
\usepackage{algorithm}
\usepackage{algorithmic}
\graphicspath{ {./images/} }
\usetikzlibrary{shapes}
\usepgfplotslibrary{polar}
\usetikzlibrary{decorations.markings}
\usetikzlibrary{backgrounds}
\pgfplotsset{every axis/.append style={
                    axis x line=middle,    % put the x axis in the middle
                    axis y line=middle,    % put the y axis in the middle
                    axis line style={<->,color=blue}, % arrows on the axis
                    xlabel={$},          % default put x on x-axis
                    ylabel={$},          % default put y on y-axis
            }}
\newcommand{\numpy}{{\tt numpy}}    % tt font for numpy
\usepackage[utf8]{inputenc}

\newtheorem{theorem}{Theorem}
\newtheorem{lemma}[theorem]{Lemma}
\newtheorem{proposition}[theorem]{Proposition}
\newtheorem{corollary}[theorem]{Corollary}
\newtheorem{conjecture}{Conjecture}
\newtheorem{definition}{Definition}
\theoremstyle{remark}
\newtheorem{example}{Example}
\newtheorem{remark}[example]{Remark}

\title{Inequalities}
\author{MinSeok Song}
\date{}
\usepackage{pgfplots}
\pgfplotsset{compat=1.18}
\begin{document}

\maketitle
\subsection*{Markov Inequality}
Start with $P(Z\geq a)\leq \frac{E(Z)}a$\\
It follows that $P(Z>1-a)\geq \frac{\mu-(1-a)}a$. The idea is that if you know that $Z$ is less than or equal to 1, then you use Markov inequality to $1-Z$; then we get the opposite inequality direction from Markov inequality. 

\subsection*{Hoeffding's inequality}
Assume that $Z_i$'s are i.i.d. samples and $P[a\leq Z_i\leq b]=1$ for every $i$. Further let us say $E(Z_i)=\mu$. Then we have $P(\bar Z-\mu)\leq 2\exp(\frac{-2m\epsilon^2}){(b-a)^2}$.
\begin{remark}
Hoeffding's inequality provides a decay rate of deviation (it is exponentially fast). 
\end{remark}

\subsection*{Central limit theorem}
Central limit theorem states that $\sqrt n(\bar{X_n}-\mu)$ converges in distribution to $\mathcal{N}(0,\sigma^2)$
\begin{remark}
CLT gives the rate of convergence of law of large number, which is $\frac{1}{\sqrt{n}}$.
\end{remark}
\end{document}