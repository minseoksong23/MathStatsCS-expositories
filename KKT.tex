\documentclass{article}
\usepackage[hyphens,spaces,obeyspaces]{url}
\usepackage{pgfplots}
\usepackage{tikz}
\usepackage{graphicx}
\graphicspath{ {./images/} }
\usepackage{amsmath}
\usepackage{amsthm}
\usepackage{amssymb}
\usepackage{mathabx}
\usepackage{amsfonts}
\usepackage{enumitem}
\graphicspath{ {./images/} }
\usetikzlibrary{shapes}
\usepgfplotslibrary{polar}
\usetikzlibrary{decorations.markings}
\usetikzlibrary{backgrounds}
\pgfplotsset{every axis/.append style={
                    axis x line=middle,    % put the x axis in the middle
                    axis y line=middle,    % put the y axis in the middle
                    axis line style={<->,color=blue}, % arrows on the axis
                    xlabel={$x$},          % default put x on x-axis
                    ylabel={$y$},          % default put y on y-axis
            }}
\newcommand{\numpy}{{\tt numpy}}    % tt font for numpy
\usepackage[utf8]{inputenc}

\newtheorem{theorem}{Theorem}
\newtheorem{lemma}[theorem]{Lemma}
\newtheorem{proposition}[theorem]{Proposition}
\newtheorem{corollary}[theorem]{Corollary}
\newtheorem{conjecture}{Conjecture}
\newtheorem{definition}{Definition}
\theoremstyle{remark}
\newtheorem{example}{Example}
\newtheorem{remark}[example]{Remark}

\title{On KKT optimality conditions}
\author{MinSeok Song}
\date{}
\usepackage{pgfplots}
\pgfplotsset{compat=1.18}
\begin{document}

\maketitle

\begin{theorem}
Assume that functions $f_0,\dots,f_m,h_1,\dots,h_p$ are differentiable. We
 are trying to solve the optimization problem

\begin{align*}
        \text{Minimize } & f_0(x) \\
        \text{subject to } & f_i(x) \leq 0, \quad i = 1, 2, \ldots, m, \\
        & h_j(x) = 0, \quad j = 1, 2, \ldots, p.
        \end{align*}
$x^*$ is a solution of this problem and the strong duality holds if and only if $x^*$ satisfies

\begin{align*}
        & f_i (x^*)\leq 0,\quad i=1,\dots,m\\
        & h_i(x^*)=0,\quad i=1,\dots ,p\\
        & \lambda_i^*(x)\geq 0,i=1,\dots,m\\
        &\lambda_i^* f_i(x^*)=0,i=1,\dots,m\\
        &\nabla f_0(x^*)+\sum^m_{i=1}\lambda_i^*\nabla f_i(x^*)+\sum^p_{i=1}\nu_i^* \nabla h_i(x^*)=0\\
        & \text{for some } \lambda^*\succeq 0
\end{align*}

\begin{proof}
        $(\Leftarrow)$ The key here is to notice that 
        $\lambda \geq 0$ because this makes $L$, a duality function, convex.
        It follows that $x^*$ gives a minimum solution of $L$ by the last condition.
        Using the fourth and the second condition, we can see that 
        $f_0(x^*)=g(\lambda^*,\nu^*)\\\\$
        $(\Rightarrow) \text{ This holds by strong duality}$. 
        \end{proof}
\end{theorem}
\begin{itemize}
\item $\lambda^T f_i(x)=0$ is called complemtary slackness. The term complementary
 comes from the idea of either or that, and slackness means "leftover." This condition is
  hard to verify in practice.
\item This is a fairly constraint condition. If the strong duality condition does not hold, then
KKT condition might not hold. Further, of course, by no means do we have existence and uniqueness.
\item There are lots of variants including nonlinear one, one of which is
 given below.
\end{itemize}
\begin{lemma}
Consider now the equality constrained problem, adding the condition\\
\begin{align*}
        & h_i(x)=0,i=1,2,\dots,p\\
\end{align*}
from the previous setup. Assume all functions are continuously differentiable. 
Let $x^*$ be the global minimum of a problem. Assume that
 the gradients of $h_i(x),i=1,2,\dots,p$ are linearly independent at $x^*$. There
  exist $\nu_1,\nu_2,\dots,\nu_p\in\mathbb{R}$ (the Lagrange multipliers) such that\\
  $\nabla f(x)+\sum^p_{i=1}\nu_i\nabla h_i(x)=0$.
\end{lemma}
\begin{remark}
\item Note that complementary slackness and linearity of $h_i$ are compensated by independence, with a 
simpler setup here - in a way, independence is a pretty strong condition here.
\item 
\end{remark}
\begin{proof}
It follows by observing that $\nabla f(x)+\sum_{i=1}^p\nu_i\nabla h_i(x)=0.$ By independence,
 we also have a uniqueness.
\end{proof}
\begin{theorem}
Assume further that we have continuously differentiable functions $g_i(x)\leq,i=1,2,\dots,m$.\\
Let us assume that the vector set composed of $\nabla h_i(x)$ and the gradients of all active inequality constraints
 form a linear independent set. At $x^*$, the KKT condition holds.
\end{theorem}
\begin{remark}
\item Gradient is the steepest direction, and independence signifies the well-behavedness of
 the constraint, in the sense that whenever $x^*$ is in the 'edge' of the inequality constraint, this direction should give a new information. The
  theorem says that this well-behavedness is sufficient condition for KKT.
\end{remark}
\begin{proof}
\begin{enumerate}
\item We are enough to show for the case $g_i$'s are all active at $x^*$, because
 $\mu_i=0$ whenever $g_i\neq 0$ since $\mu_i\geq 0$ (otherwise we do not have minimum at $x^*$).
\item Since $g_i\leq 0$, we can also assume that inequality is equality, 
and use the previous theorem to show the existence of Lagrangian constants that satisfy $\nabla L=\nabla_x f(x^*)+\sum^m_{i=1}\mu_i\nabla_x g_i(x^*)+\sum^p_{i=1}\nu_i\nabla_x h_i(x^*)=0$.
\item Now it suffices to prove $\mu_i\geq 0$ for all $i$'s. Assume, to the contrary, that $\mu_i<0$ for some $i$.
First note that $\sum^m_{i=1}\mu_i\nabla_x g_i(x^*)+\sum^p_{i=1}\nu_i\nabla_x h_i(x^*)=-\nabla_x f(x^*)=0$ since $x^*$ achieves minimum. 
Use constant rank theorem on $F(t,x)=(g_1(x),\dots t+g_i(x),\dots, g_m(x),h_1(x)\dots,h_p(x))$ (we may need additional condition, such as independence of gradients). 
We can derive the contradiction by having $\nabla_x f(x^*)=-\sum^m_{i=1}\mu_i\nabla_xg_i(x^*)-\sum^p_{i=1}\nu_i\nabla_x h_i(x^*)\neq 0$.
\end{enumerate}
\end{proof}
\begin{remark}
\item Dual is not necessarily symmetric in mathematics in general. For example the dual of $L^1$ is $L^\infty$ but the dual of $L^\infty$ is not $L^1$.
\item Under certain condition, it is; for example if $f$ is convex and closed, (epigraph is a closed set) then $f^{**}=f$.
\item Another relevant concept for convex analysis (in particular, for function) is Legendre transform.
\end{remark}
\end{document}
