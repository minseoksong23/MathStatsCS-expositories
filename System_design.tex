\documentclass[11pt,reqno]{amsart}
\usepackage{graphicx, url}
\usepackage{enumerate}
\usepackage{mathtools}
\usepackage[margin=1in]{geometry}
\usepackage{graphicx}
\usepackage{pgfplots}
\usepackage{tikz}
\usepackage{graphicx}
\usepackage{enumerate}
\usepackage{titlesec}
\graphicspath{ {./images/} }
\usepackage{amsmath}
\usepackage{amsthm}
\usepackage{amssymb}
\usepackage{mathabx}
\usepackage{amsfonts}
\usepackage{enumitem}
\usepackage{algorithm}
\usepackage{algorithmic}
\graphicspath{ {./images/} }
\usetikzlibrary{shapes}
\usepgfplotslibrary{polar}
\usetikzlibrary{decorations.markings}
\usetikzlibrary{backgrounds}
\pgfplotsset{every axis/.append style={
                    axis x line=middle,    % put the x axis in the middle
                    axis y line=middle,    % put the y axis in the middle
                    axis line style={<->,color=blue}, % arrows on the axis
                    xlabel={$},          % default put x on x-axis
                    ylabel={$},          % default put y on y-axis
            }}
\newcommand{\numpy}{{\tt numpy}}    % tt font for numpy
\usepackage[utf8]{inputenc}

\newtheorem{theorem}{Theorem}
\newtheorem{lemma}[theorem]{Lemma}
\newtheorem{proposition}[theorem]{Proposition}
\newtheorem{corollary}[theorem]{Corollary}
\newtheorem{conjecture}{Conjecture}
\newtheorem{definition}{Definition}
\theoremstyle{remark}
\newtheorem{example}{Example}
\newtheorem{remark}[example]{Remark}

\titleformat{\section}
  {\normalfont\Large\bfseries\centering}{\thesection}{1em}{}



\title{A Programmer's perspective}
\author{MinSeok Song}
\date{}
\usepackage{pgfplots}
\pgfplotsset{compat=1.18}
\begin{document}

\maketitle

\section*{Code Security}
\begin{itemize}
    \item In the beginning of the chapter, B $\&$ H introduces two scenarios regarding vulnerability.
    \item 1) Unlike Java, C doesn't feature a garbage collector. However, when declaring an array such as char kbuf[KSIZE];, the array kbuf can be initialized with garbage values. 
    In two's complement representation, there's a significant difference between -MSIZE and MSIZE due 
    to the most significant bit being set to 1 for negative numbers. This discrepancy can introduce vulnerabilities and potentially lead to memory leaks.
    \item 2)If we do malloc(ele\_cnt * ele\_size) but if each argumnent is very large, the program might not allocate the desired memory size due to integer overflow.
\end{itemize}

\section*{Two's Complement}
\subsection*{arithmetic and Underflow/Overflow}
When adding two numbers in two's complement representation, there's a potential for underflow and overflow. \\
Let's denote the result of addition as \(TAdd_w(u,v)\) \\ where \(w\) is the bit width of the numbers, and \(u\) and \(v\) 
are the numbers being added. Then, we can define \(TAdd_w(u,v)\) as:

\[
TAdd_w(u,v) = 
\begin{cases} 
    u + v + 2^w & \text{if } u + v < TMin_w \\
    u + v & \text{if } TMin_w \leq u + v \leq TMax_w \\
    u + v - 2^w & \text{if } u + v > TMax_w
\end{cases}
\]

Here, \(TMin_w\) and \(TMax_w\) are the minimum and maximum values representable with \(w\) bits in two's complement, respectively.

For instance:
\begin{itemize}
    \item When we add two values (i.e., 0\ldots and 0\ldots) and get a result starting with 01\ldots, it indicates an overflow (desired: $a-2^{w-1}+b-2^{w-1}=a+b-2^w$, but what we get: $a+b+2^w-2^w=a+b$, where $a$ and $b$ are bits after the most significant bit, so subtract $2^w$).
    \item When we add two values (i.e., 1\ldots and 1\ldots), and the result is 1\ldots, it indicates an overflow (desired: $y$, but what we get: $-(2^w-y)=y-2^w$, so add back $2^w$).
\end{itemize}
\begin{remark}
\begin{itemize}
\item Multiplication and division are a bit more involved; but essentially, we use addition and bit-shifting operations to accomplish tasks (we need to be careful when dividend is negative number and introduce the concept of "bias"). 
\item When sign-extending an integer using the $>>$ (right shift) operator, the most significant bit (often called the sign bit) is replicated to fill in the shifted positions.
\end{itemize}
\end{remark}

\section*{Assembly Language}
\subsection*{Register for x86-64}
\begin{itemize}
\item Extensive list is on \url{https://wiki.osdev.org/CPU_Registers_x86-64}
\item RIP: Instruction pointer(used for count)
\item RAX, RBX, RCX, RDX: general Purpose Registers, with RAX often used for return values.
\item R8 to R15: extra general purpose register
\item RSP: Stack pointer, RBP: Base Pointer
\item RDI, RSI, RDX, RCX, R8, R9: used for function arguments
\end{itemize}
\subsection*{Basics, Control Flow}
\begin{itemize}
\item The type of processor (specifically its architectore or ISA - instruction Set Architecture) dictates the set
 of instructions that are available for use.
\item The example of processor includes
\begin{enumerate}
\item Intel: x86(widely-used 32-bit architecture), IA32(Intel's 32-bit architecture), \\
Itanium(64-bit architecture developed by Intel, didn't see wide adoption compared to\\
 x86-64 pioneered by AMD), x86-64(later cross-licensed, enabling both companies to introduce enhancements since)
\item ARM: Used in almost all smartphones
\end{enumerate}
\item There are three ways to get assembly language for the code.
\begin{enumerate}
\item Use 'gcc' with the '-S' flag. For example, put gcc -S source.c
\item Reverse engineering for the compiled binary. For example, put objdump -d binary\_name
\item Using 'gdb' debugger. For example, go to gdb and put disassemble function\_name
\end{enumerate}
\item The last two options are appropriate when you do not want to go into the source code directly.
\item Address computation
\begin{itemize}
\item 0x8 (\%rdx) means to add 0x8 to the address \%rdx.
\item (\%rdx, \%rcx) means to add the address \%rdx with \%rcx.
\item (\%rdx,\%rcx,4) means to add the address of \%rdx with four times the address of \%rcx.
\end{itemize}
\item Some instructions for x86-64 architecture
\begin{enumerate}
\item movq: simply just moving; q stands for quadword (64bits, versus 8 bits for byte, 16 bits for word, and so on), meaning that this operates on a 64-bit quadword operand.
\begin{example}
When we do movq (\%rsi), \%rdx, this means that we are assigning the value at the address \%rsi to the register \%rdx.
\end{example}
\item leaq: this stands for "load effective address quadword."
\begin{example}
when we do leaq (\%rbx, \%rcx, 4), \%rax, we'd calculate \%rbx+4*\%rcx and put this address into \%rax. This is
 different when using movq, we would access the value at that address.\\
 Note that \%rbx and \%rcx can be address or register.
\end{example}
\item addq: add, imulq: multiplication, salq: shift, ret: return, etc
\item when we do cmp, add, sub, etc, some flags (which belong to FLAGS register) are implicitly set, so we can use them for control flow.
\item when we do cmp Src2, Src1, we effectly do Src1-Src2 but does not store result (as opposed to sub).
\begin{enumerate}
\item CF(carry flag) set: if carry out from most significant bit 
\item ZF(zero flag) set: if a == b
\item SF(sign flag) set: if (a-b) < 0
\item OF(overflow flag) set: if two's complement overflow: $(a>0 \& b<0 \& (a-b)<0) || (a<0 \& b>0 \& (a-b)>0)$
\end{enumerate}
\item some notable registers when we use control flow: temporary data, location of runtime stack
, location of current code control point(points to the instruction being executed), status of recent tests.
\item when we say \%al, \%bl, \%cl, and so on, it refers to the lower end of \%rax, \%rbx, \%rcx, and so on.
\item movzbl: stands for move. zero-extend, byte, long. Extend the rest 32 bits to zero.
\item store the return value on \%rax.
\end{enumerate}
\item Conditional move vs. branch move
\begin{enumerate}
\item Conditional moves eliminate the need for branching by selecting a result based on a condition without branching.
\item If the prediction of branch move is wrong, it can lead to a pipeline stall due to mispredicted branches.
\item Even if we write if-else statement in C code, the compiler may use conditional move for optimization.
\begin{example}
pitfalls: $val = x>0 ? x*=7 : x+=3;$
\end{example}
\end{enumerate}
\item GCC has some compiler optimization levels, such as -O1, -O2, -Og (the latter being high order, hence abstracting out),etc.
 -Og is suited the most for illustration purposes. 
\begin{example}
\item gcc -Og -o p p1.c p2.c
\item This does mean that we optimize using -Og flag, and then save it as p. We compiled 
the source files p1.c and p2.c into assembly language and linked them together. 
\end{example}
\item As a result of this optimization, two different source codes can generate the same assembly code.
\item This is why reverse engineering can be challenging.
\item Control flow in assembly language can be largely summarized by the following three principles.
\begin{enumerate}
\item Conditional jump
\begin{example}
\item CMP AX, BX; Compare the contents of registers AX and BX
\item JE  Label; Jump to 'Label' if AX equals BX (JE stands for "Jump if Equal")
\end{example}
\item Conditional move
\begin{example}
\item CMP AX, BX; Compare the contents of registers AX and BX
\item CMOVNE CX, DX; Move the value from DX to CX only if AX is not equal to BX
\end{example}
\item Indirect jump via jump tables
\begin{example}
\item movq \%rdx, \%rcx
\item cmpq \$6, \%rdi \# x:6
\item ja      .L8
\item jmp \*.L4(,\%rdi, 8)
\end{example}
\end{enumerate}
\end{itemize}
\subsection*{Machine Procedures}
\begin{itemize}
\item ABI: a set of conventions about the register.
\item There are three aspects: passing control, passing data, and memory management.
\item Passing control
\begin{enumerate}
\item 'call' instruction is used to initiate a subroutine; addresses control.
\item \%rsp stores the address of the top of the call stack.
\item When a function is called, the return address (the address of the instruction immediately 
following the function call) is pushed onto the stack. 
\end{enumerate}
\item Passing data
\begin{enumerate}
\item We pass the first six data onto register \%rdi, \%rsi, .., \%r9, (in this order, by convention)
 and move over to stack after sixth argument.
 \item Return value is stored in \%rax.
\end{enumerate}
\item Memory management
\begin{enumerate}
\item The usage of \%rbp is optional - we usually know how much we "step back" by the amount of memories we need to deallocate.
\end{enumerate}
\item There are two register saving conventions
\begin{enumerate}
\item Caller-Saved Registers: Before making a function call, the caller 
is responsible for saving the current values of these registers 
if they need to be preserved. After the function call returns, 
the caller can restore the values if necessary. Examples 
of caller-saved registers include: \%rax, \%rdi, \%rsi, ..., \%r11.

\item Callee-Saved Registers: When a function is called (i.e., 
the callee), it's the function's responsibility to save the current 
values of these registers if it intends to
 use them. Before returning, 
the function restores their original values. Examples of callee-saved 
registers are: \%rbx, \%r12, \%r13, \%r14, \%rbp, \%rsp.
\end{enumerate}
\end{itemize}

\end{document}